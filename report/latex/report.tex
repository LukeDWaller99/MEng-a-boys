\documentclass [12pt]{article}
\usepackage{fancyhdr}
\usepackage[utf8]{inputenc}
%\usepackage[english]{babel}
\usepackage{gensymb}
\usepackage{amsmath,amssymb,amsfonts}
\usepackage{algorithmic}
\usepackage{graphicx}
\usepackage{textcomp}
\usepackage[table]{xcolor}
\usepackage{siunitx}
\usepackage{indentfirst}
\usepackage[margin=2.5cm]{geometry}
\usepackage{float}
\usepackage{hyperref}
\usepackage[UKenglish]{datetime}
\usepackage{setspace}
\usepackage{hhline}
\usepackage{amssymb}
\usepackage{lastpage}
\usepackage[acronym]{glossaries}
\usepackage{listings}
\usepackage{color}
\usepackage{booktabs}% http://ctan.org/pkg/booktabs
\newcommand{\tabitem}{~~\llap{\textbullet}~~}
\usepackage{subfigure}
\usepackage[T1]{fontenc}
\usepackage[numbers]{natbib}
\usepackage{helvet}
\renewcommand{\familydefault}{\sfdefault}

\definecolor{customgreen}{rgb}{0,0.6,0}
\definecolor{customgray}{rgb}{0.5,0.5,0.5}
\definecolor{custommauve}{rgb}{0.6,0,0.8}

\makeglossaries
\loadglsentries{glossary.tex}

\graphicspath{{images/}}

\hypersetup
{
    colorlinks=true,
    linkcolor=black,
    urlcolor=blue,
    citecolor=black,
}

\doublespacing

\title{}
\author{}
\date{\today}

\begin{document}

\begin{titlepage}
    \begin{center}
        \vspace*{1cm}

        \vspace*{\baselineskip}

        \rule{\textwidth}{1.6pt}\vspace*{-\baselineskip}\vspace*{2pt}
        \rule{\textwidth}{0.4pt}\\

        {\LARGE EXAMPLE TEXT}\\[0.2\baselineskip]

        \rule{\textwidth}{0.4pt}\vspace*{-\baselineskip}\vspace{3.2pt}
        \rule{\textwidth}{1.6pt}\\[\baselineskip]
        \scshape

        \textbf{Student No: 10618407}

        \today

        \vfill

        \vspace{0.8cm}

        \includegraphics[width=0.4\textwidth]{UOP_Logo.png}

    \end{center}
\end{titlepage}

\newpage
\fancyfoot[R]{Page \thepage \hspace{1pt} of \pageref{LastPage}}
\pagenumbering{roman}
\tableofcontents
\listoffigures
\listoftables
\printglossaries

\newpage
\begin{abstract}

\end{abstract}
\section{Acknowledgements}

\newpage
\pagenumbering{arabic}
\section{Introduction}
\setcounter{page}{1}

\newpage
\section{Literature Review and Prior Art}


\newpage
\section{Business Justifications}

\subsection{Research Techniques and MECE Analysis}

The team engaged in innovation talks throughout the project, particularly with 42Technology \cite{42T}, to understand the framework commonly used to solve complicated problems within industry. A particular method used involved the definition of the problem broken down into mutually exclusive sub-problems that are collectively exhaustive. This method is referred to as \gls{mece} Analysis and enabled the team to focus on various elements of the project to achieve one shared goal, without overlapping or reproducing any work. 

This method aided the understanding project criteria from the sponsor’s perspective regarding patenting, and from the university’s perspective of academic evaluation. The PEP defined the market gap, drawing from research gathered that detailed the lack of EV charging infrastructure within the UK, and how realistically, the public cannot rely on shared charging points. However, \gls{mece} analysis has uncovered the \gls{hevcs} as a feasible solution to such a challenge, and that the situation would benefit from having a mobile, versatile approach to suit the varied requirements of \gls{ev} users in different locations around the UK. 

\subsection{Addressing the Key Challenges of Electric Vehicle Adoption}

The common barriers preventing the public from purchasing EVs being price, range and charging infrastructure. The \gls{hevcs} tackles all three of these problems, and the team have designed the platform and its use specifically for its application within the real-world. 

\subsubsection{Charging Infrastructure and Trends in Usage}

Currently, the UK Government’s National Traffic Survey stated that 938,182 electric cars were registered to be on the roads as of Q3, 2022 \cite{Q32022}. Comparatively, a survey published in October 2022 stated that 34,637 public charging points were currently installed \cite{chargestats}. These figures indicate that the charging infrastructure available cannot keep up with the current number of \gls{ev}s on the road and raises concerns as this number is set to rise with the 2030 ban on selling new petrol and diesel vehicles. Furthermore, it is predicated that 99\% of miles will come from \gls{ev}s by 2050, a figure that has been adjusted after an underestimation of their adoption rate \cite{2050rates}. 

Approximately 50\% of UK charging points are placed at end of journey destinations, typically within car parks the public frequent for shopping, leisure, and down-time activities \cite{chargestats}. A Government survey revealed a majority's preference to charge \gls{ev}s at home, with 66\% and of \gls{bev} and 41\% of \gls{phev} users had their own dedicated charging point within a private driveway or garage \cite{homecharge}. The survey highlights that users felt dedicated charging points were either too expensive or complex to install.

Conversely, 53\% of the UK do not have a driveway or garage that would allow for the possibility of installing a domestic charging point. This has proven to be a deciding factor in the public adoption of \gls{ev}s. The portion of the market purchasing \gls{bev}s \ \gls{phev}s using public dedicated charging points within the vicinity of their residence is 4\% and 1\% respectively \cite{chargestats}. Thusly, a market gap to provide a product that can charge EVs within the immediate surroundings of the home has been identified. Moreover, the \gls{hevcs} does not require a dedicated parking space, and can be used at the individual's discretion. 

\subsubsection{National Grid Demands}

When considering the typical commuting patterns, full-time workers with typical 9-5 jobs are amongst those susceptible to aligning their arrival to domesticated regions. 60\% of EV users prefer to charge between 5-8pm, meaning the national grid struggles with the demand of those coming home, with up to 20\% of the UK's EV fleet charging between those hours \cite{chargestats}. The \gls{hevcs} could charge during the off-peak times, lessening the load on the grid.

\subsubsection{Battery Range and Efficiency}

The National Travel Survey presented statistics in 2021 stating that 98.5\% of journeys taken in the UK were less than 50 miles \cite{nts}. The capacity of the payload compartment was calculated through an analysis of both power consumption and efficiency, conducted using data from the UK public’s current \gls{ev} usage trends. The average efficiency determines that 1 kWh equates to approximately 3.5 miles, and second-hand EV batteries typically have a capacity of 80-90\% of the original value \cite{strickland2014estimation}. The \gls{hevcs}’s configurable payload compartment can supply the user with 10 to 50 miles of range, depending on the \gls{ev} batteries used, as shown in the research and calculations section.

\subsubsection{The Cost of Charging an Electric Vehicle}
The \gls{hevcs} presents itself as a cost-effective product that reduces the price of charging an \gls{ev}. Within the UK, domestic electricity prices have been on the rise, and the cost of charging at public points has increased. Furthermore, the spaces in which to charge are not guaranteed, a free to park still must be paid. A cost analysis presents the additional benefits of an individual having their own \gls{ev} charging device, such as the \gls{hevcs}, that remains domestically powered.

Providers, such as EDF Energy \cite{edf}, offer \gls{ev} charging tariff that allow the user to charge their vehicle for 9p per kWh during off-peak times, oppose to 46p per kWh at public charging points within Plymouth. The typical UK driver makes 14 trips a week, each with an average distance of 8.4 miles, equating to an annual mileage of $ \approx 6,000 $ miles. This results in a running cost of £788 for vehicles with an efficiency of $ \approx 3.5 $ charging on public tariffs. By using the reduced domestic tariff, the user would spend £154 a year, the savings of which would accumulate to cover the price of the \gls{hevcs} platform over a span of several years.

\subsubsection{Environmental Benefits}
The \gls{hevcs} can give a second life to \gls{ev} batteries, while educating users on the importance of understanding their daily commute, and what it demands in terms of power. In addition, it is understood that frequently charging \gls{ev} batteries to their maximum capacity, and letting them discharge fully, reduces their lifespan \cite{haram2021feasibility}. This device can bridge the learning gap, educating users on what it means to operate a battery within its ideal state of charge range to improve longevity. 
It is common knowledge that EV batteries are designed to possess an ability to retain their original battery capacity over many years, but its ability to do so depends on a variety of factors. Charging schedules that keep the state of charge between 20-80\% would provide an adequate amount of power to fulfil the average daily commute. Moreover, the \gls{hevcs} would allow the user to take advantage of off-peak electricity rates, smart charging features, and monitoring energy consumption. Through adopting a maintenance charge approach, following the natural patterns to meet the user’s everyday needs, the user can identify how to adjust their habits to positively impact batter health and costs.


\newpage
\section{Requirements}

\newpage
\section{Optioneering}

\subsection{Hardware Justifications}

\subsubsection{Microcontroller Selection and IDE}
STM32 boards were chosen as they provide scalability in terms of hardware resources and software development tools through the use of mbed. With a wide range of microcontroller options available, the team were able to choose the STM32 board that best suits the project requirements, such as processing power, memory capacity, and peripheral integration. This scalability enables the testing of software on different STM32 platforms, ensuring compatibility and performance across all three platforms.

STM32 boards are known for their robustness and reliability. The microcontrollers are designed with built-in features that enhance the reliability of software, such as memory protection units, error correction codes, and fault detection mechanisms. These features contributed to the stable and dependable operation of the platform, reducing the likelihood of software failures and improving overall system reliability. Furthermore, the STM32 boards used are  compatible with various \gls{rtos}, providing scheduling and multitasking capabilities. This allowed the team to test and evaluate the software's behaviour under real-time constraints. This enables the validation of time-critical functionalities, responsiveness, and task management, ensuring reliable and deterministic system operation. Finally, the boards offer excellent integration capabilities with external testing equipment and tools, such as debuggers and pico-scopes. This integration enables comprehensive hardware and software debugging, performance monitoring, and analysis, facilitating in-depth testing and troubleshooting. 

\subsubsection{F429ZI}
This section provides an analytical comparison of the STM32F429ZI microcontroller against alternative options for embedded systems requiring reliability, multithreading capabilities, and diverse communication interfaces. It investigates the suitability of the STM32F429ZI microcontroller for interfacing with switches, joysticks, \gls{vesc}s, and servos. By considering its technical specifications, architectural features, and development ecosystem, the decision to use this board was backed up by its various abilities.  

Embedded systems play a crucial role in modern technological applications, encompassing a wide range of industries such as robotics, automotive, and industrial automation. The selection of an appropriate microcontroller for a given application is paramount to ensure reliability, efficient multitasking, and seamless communication with external devices. In this regard, the STM32F429ZI microcontroller presents a compelling option due to its exceptional features and capabilities. It is built upon the high-performance ARM Cortex-M4 core. Its rich set of peripherals and advanced architecture make it a prime choice for complex embedded systems. Noteworthy specifications include a clock speed of up to 180 MHz, ample flash memory and \gls{ram}, and a versatile set of communication interfaces, such as \gls{uart}, \gls{spi}, I2C, and USB. 

Multithreading is a critical aspect of this project as it requires efficient execution of concurrent tasks. The STM32F429ZI microcontroller supports multithreading through its advanced interrupt handling mechanism and hardware-based multithreading support. This feature allows for the seamless execution of multiple tasks simultaneously, enhancing overall system performance and responsiveness. Furthermore, reliability is a key requirement in mission-critical applications. The STM32F429ZI microcontroller excels in this aspect by incorporating error correction codes (ECC) in its memory system, ensuring data integrity and fault tolerance. Additionally, its robust power management features, low-power modes, and built-in watchdog timers contribute to the overall system's reliability and fault resilience. 

The ability to communicate with various external devices is an essential requirement of this project. The STM32F429ZI board offers a multitude of communication interfaces, allowing seamless integration with controller peripherals, as well as the \gls{vesc}s and servos. The flexibility regarding \gls{gpio} pins, \gls{pwm} outputs, and dedicated hardware interfaces provide the necessary capabilities to interface with these devices efficiently. The use of \gls{api}s for which are detailed in the official STM32 documentation. These include reference manuals, datasheets, and application notes, providing valuable information on hardware features, software development guidelines, and testing methodologies. The platform offers an extensive collection of pre-built libraries, encompassing diverse functionalities such as communication protocols, sensor interfaces, motor control, and more. Leveraging these libraries not only saves significant development time but also facilitates the rapid prototyping of complex projects, such as the \gls{hevcs}. Consequently, pairing these factors with the team’s familiarity with ARM Cortex-M based microcontrollers, development with the STM32F429ZI was intuitive in nature and accelerated the embedded system’s functionality. 

\subsubsection{L432KC}
The \gls{hevcs} controller required a compact microcontroller to integrate an \gls{lcd} display, D-pad buttons, and \gls{spi} communication. The STM32L432KC controller’s technical specifications, architectural attributes, and performance advantages met the requirements of the targeted features. Furthermore, the board is recognised for its low power consumption, operation efficiency, and its use within portable applications. The \gls{lcd} operates using hardware accelerated graphical features, such as DMA-driven memory transfers. The \gls{gpio} pins have multiple configurations and include external interrupt controllers facilitate the D-pad input processing. The STM32L432KC includes multiple \gls{spi} peripherals that enable high-speed data transfers over the \gls{spi} bus, minimising CPU overhead.  

The STM32L432KC microcontroller outperforms most Arduino boards in terms of processing power, thanks to its advanced ARM Cortex-M4 core. The STM32L432KC's high clock speed enable efficient execution of complex algorithms, making it suitable for applications that demand real-time processing, data manipulation, and signal analysis. Unlike Arduino, the STM32 family allows developers to delve into low-level programming, giving them greater control over hardware resources. The microcontroller supports the C/C++ programming language, and developers can directly access registers and peripherals, optimizing performance and resource utilization. The team believe this level of customizability is particularly beneficial for this project, following from research into applications that require fine-grained control over system behaviour. 

\subsubsection{ESP-8266}
Within this project, the ESP-8266 presents itself as a low-budget option capable of reliable data saving capabilities using Wi-Fi. It has a compact design with a vast range of peripherals available. Its highly integrated System-on-a-Chip (SoC) architecture reduces requirements for additional components. Its enhanced power management features enable the ESP to be used during periods of long-term operation on a limited power source. Through examination of the features, its network connectivity was enhanced within the project. The ESP-8266’s ability to connect to the internet using Wi-Fi opens a plethora of opportunities for \gls{iot} applications. The support provided includes various options regarding networking protocols, with detailed guides on how to implement each.

\subsubsection{Super500 Servos}

\subsubsection{6384 Motors}

\subsection{Software Justifications}

\subsubsection{MicroPython}

\subsubsection{C++ and Object Orientated Programming}
As C++ is a widely supported language, it’s application and portability amongst various platforms and controllers is greatly enhanced. The team were able to migrate their applications between platforms with ease, saving time and increasing efficiency. A range of libraries were used on both the F429ZI and L432KC, along with mbed \gls{api}s that did not require significant modification between platforms. The team recognised the decision to use C++ as an essential requirement due to the importance of optimisation and reliable communication protocols.  

Using this language allowed the team to control various signals directly, without abstraction, making the system easy to debug and implementation of driving signals easier to deploy. The team felt that this was an important aspect of creating a prototype in a fast-paced environment, where multiple stages of embedded system development were required. Furthermore, the team had experience with \gls{oop}, and appreciate the structure that classes provide as building blocks of an embedded system. Data validation and constraints were tested to determine that only valid data and correct values were accepted by the system, maintain integrity and reliability of the data stored. This was particularly useful as the team could deploy classes modified through inheritance that had already been fully tested, reducing the likelihood of errors. In terms of quality assurance, OOP ensured modularity and organisation of the system and its interactions with both isolated classes and those conducting shared data-related operations.  

Upon reflection, the team felt that their approach reduced complexity within the system, and enabled others to continue work due to the organised structure. Particularly when passing signals through various communication methods and channels, where the correctness of the data was vital. The team were able to establish the correct behaviour of each class and efficiently isolate problems throughout the development of the project. 


\subsubsection{VESC Software}
The VESC Tool, a highly versatile and widely used application, was used to configure, and program the \gls{vesc}s. Similarly, this tool is used to configure electric skateboards, bikes, robots, and electric vehicle applications. The user-friendly, intuitive interface guides the user through adjusting various parameters. These include customised motor control parameters such as acceleration, braking profiles, maximum motor current, battery voltage limits, throttle curves and so forth. The \gls{vesc} Tool also provides real-time monitoring information, while logging such data to create a performance record, to display visualizations of relevant parameters. Furthermore, the advanced features include regenerative braking and \gls{foc}. Also known as Vector or Direct Torque Control, \gls{foc} allows for precise control of both speed and torque of the motors.  

\subsection{Manufacturing Justifications}

The first prototype was built from extrusion and 3D printed PLA. The PLA was great for testing an unloaded chassis, but it could not bear additional weight. The team explored various designs to share the load and increase the structural integrity of the wheel mount, but they were unsuccessful. A group decision was made to design a simple motor drive system that could be contained within a small space and manufactured quickly and efficiently.  

The aluminium extrusion was cut by hand using a hacksaw, then filed. This was deemed as an accurate method to use for making straight cuts and provided the degree of accuracy needed. Both the extrusion and aluminium sheets were recycled from the remains of other projects. The motor drive assembly plates were created from laser cut 4mm thick aluminium sheets. A laser cutter was used because multiple detailed cuts were required to assemble four separate motor drivers. This method was time efficient, resulted in precise cuts, and did not compromise the structural integrity of the sheets themselves.  

Grub screw holes were added to the pulleys with a pillar drill due to its stability and control related accuracy. A convenient option as the equipment was readily available in the university’s workshop, and it enabled the team to purchase cheaper components that required small alterations. The pillar drill produced consistent results through the action of repetitive alterations. 

Both hand saws and bench top saws were used for creating motor driver supports and servo harnesses from scrap wood. They were used to attach the supports directly to extrusion and to replace stand-offs by spreading the force over a larger area. The use of recycled wood was appropriate for the application, meaning the team was happy with its performance.  

\newpage
\section{Hardware Design}

\newpage
\section{Software Design}

\newpage
\section{Testing}

\newpage
\section{Evaluation}

\subsection{Leadership Methodology}
The team adopted a democratic approach, and although various ideas to solve problems were explored, there were no disagreements during development. Individual team members worked on tasks they had relevant experience in, enabling the sharing of knowledge while facilitating fast development. However, to honour the NDA, the team had to work in a separate room to the rest of the cohort. This was beneficial at times but resulted in less outside perspectives from university colleagues. This was recognised by the team as they chose to remain objectively critical and seek feedback from those deemed appropriate. The team acknowledge that the way in which they operated as a team was effective, but upon reflection wish that more academic feedback was sought throughout. 

\subsection{Project Progression}
The project’s development reflected the GANTT chart up until the middle of March. At this point, the team changed the design to make it more modular, requiring further testing and a redesign of the chassis, motor, and servo mounts. The team do not regret this change to the design as it greatly increased the platform’s performance and functionality. This did, however, have an impact on the aesthetics of the platform as the outer casing manufacturing was delayed.  The team discussed this and determined that functionality was more important, and that the overall structural integrity did not rely on the casing. Furthermore, the way in which the project progressed through the defined stages of development was assessed through \gls{kpi}s and tests indicative of successful completion of a milestone. The team felt that combining this with the agile methodology meant that no developmental steps were missed, as everything was regularly reviewed directly from the critical development pathway.  

\subsection{Corporate Sponsorships}

\subsubsection{Sirmon Industries}
James and Heather Sirmon attended on-site meetings every week. Heather was critical of the project, pushing the team to consider the wider application of the \gls{hevcs} as a real-world product. She questioned our decisions, while James queried our justifications as he completed the patent application. Additionally, their regular reviews enabled the team to discuss the progress and decisions in detail, meaning that Sirmon Industries Limited were satisfied as a customer. The team feel that they completed the project in a professional manner and are glad to continue work with both Sirmon and the company that further develop the \gls{hevcs}. 

\subsubsection{European Social Fund}
Jo Byrne was available to discuss the commercial viability of the \gls{hevcs} during its development. The team often reflected on the capabilities of the project in relation to legislation, as well as defining the novel aspects that Sirmon required for the patent application. The insight provided by the \gls{esf} was vital for understanding the customer usability of such a device, and how the outside opinion of those who do not have the technological understanding can provide interesting perspectives from a customer’s point of view.  

Two reports were produced for the ESF, each covering various aspects of the project requirements and the methodology. This meant that at the start of the project each team member did a skill set review that was then used to determine who would develop what specific elements of the \gls{hevcs}. Halfway through the project, the final ESF report was due. It detailed the roles that the members of the team grew into, the milestones that had been completed, test methods used, and the business recommendations for Sirmon Industries. This helped the team to gain an insight into how the project could be developed further, while defining the scope for the \gls{hevcs}’s development as a university project.  

\subsection{Budget Management}

The university provided £1,200 in budget, and Sirmon Industries allotted £2,000. The team decided to create a bill of materials for the entire project to coordinate arrival times, costs, and availability. As the university had an approved list of suppliers, the products that were not on this list were automatically sifted to the \gls{bom} to be discussed with Sirmon Industries. It is worth noting that the components purchased were still deemed appropriate for the project, and therefore approved by the university academics. The external supplier \gls{bom} also included products with a long lead time, that James Sirmon was able to source from his industry partners instead. These decisions were beneficial to the project’s initial development, and the team believe they made the correct choice by asking Sirmon to obtain such components during the Christmas break, a period in which the university could not place or receive orders.  

Portions of the university budget were allocated for electronic components, manufacturing costs, and materials. This included the PCB component orders from approved suppliers that the university had existing free, fast shipping subscriptions with. The team felt that this decision was correct, enabling fast acquisition of vital components for chassis, motor, and hardware assembly too.  

A complete breakdown of the costings can be found in the appendix. This includes a breakdown of how £1,163 of the university budget was spent, and how Sirmon spent £1,862.76. The remaining balance was £174.25. The team acknowledge that a large proportion of the costs, equating to 500 came from purchasing \gls{vesc}s. Designing a speed controller for this project would have been time-consuming, whereas a fast, reliable, and sufficiently documented solution was required. Thusly, the project benefited from using the \gls{vesc}s, their convenience and their positive impact on the initial platform control testing.

In conclusion, the overall cost of this project far exceeded the budget provided by the university, and so the team would like to acknowledge the contributions from Sirmon Industries and extend their gratitude. It is through their generosity that the project was able to progress, possess high quality products, and operate effectively.  

\subsection{Customer Usability}

The \gls{hevcs} is reliably controlled by the user, using methods that enhance its dependability. The purpose behind the prototype is to prove its potential as an alternative method to EV charging, using methods and approaches that the user can rely on and independently control. Using the results of the \gls{mece} analysis that define \gls{ev} owner’s needs to a dependable solution to charging at the convenience, the \gls{hevcs} has proven itself as a viable alternative. Moreover, its creation presents a viable alternative to getting power to the vehicles, provided they are within a certain proximity of the home, that is not limited by obstacles such as kerbs.   

This paper has detailed how the cost of charging is decreased using off-peak domestic tariffs, enabling the user to effectively control the price they pay to charge their EV using electricity schemes. Giving the user authority over when, how, and how much they charge their vehicle increased the accessibility of owning an EV.  

\subsection{Ethics}

The ethical conduct principles defined in the PEP have been applied throughout the development of the \gls{hevcs}. A deeper understanding of the operational safety of the platform has been acquired through testing, and several efforts have been made to limit the risk of injury to both users and the public through the deployment of several safety features discussed in the design section. Namely, the magnetic breakaway cable and the object avoidance system. The team have made an effort to create safety systems that enhance human capabilities without revoking their control. For example, users can drive the platform towards a wall, and the object avoidance system will prevent them from making contact with said wall, but it does not take away their ability to continue to operate the platform in another direction that does not cause harm. Furthermore, the platform is not autonomous, as to abide by the relevant legislation, and so the public are ensured that humans retain all control of such a device.  

The research conducted throughout this project has included the use of and reference to data from national surveys, from which the participants have agreed to the distribution of their information. The team have not gathered or collected information themselves nor obtained any data through the university.   

\subsection{Environmental Impact}

The environmental impact of the components and products used to build the \gls{hevcs} have been discussed in their relevant sections. Furthermore, the future improvements include discussion of alternative methods of acquiring power through sustainable, green energy sources. The team would like to reiterate the importance of recycling \gls{ev} batteries, building on the explanation on the impact on the environment discussed in the \gls{pep}. 

\subsection{Design}

The team is pleased with the overall design of the \gls{hevcs} prototype platform and its capabilities, as the maximum height of the platform exceeded the target of 150mm. With a final height of 110mm, the \gls{hevcs} is able to fit under all bar one of the types of registered electric vehicles in UK \cite{carsinit}. Due to the confidential requirements in place for the project, the design was not able to be tested outdoors. However, the team is confident that with suitable levels of testing and a small amount of modification the \gls{hevcs} would be fully operational outdoors, on both pavement and road surfaces.  

During testing, the craft demonstrated stability and the ability to support a payload of up to 80kg before collapsing, aligning with the requirements of the project. As well as this, the craft fully abides by all relevant class 2 mobility scooter legislation, allowing it to be used on pavements.  Furthermore, preliminary testing strongly suggests that the platform has the potential to navigate and manoeuvre over kerbs, of up to 150mm, with further developments and refinements. This promising outcome bodes well for the future optimisation and further developments of the platforms design capabilities.  

Overall, the team is satisfied with the design prototype. This stems from its successful and complete adherence to the requirements of stability, wight capacity, and also for its potential developments in kerb climbing ability.  

\newpage
\section{Further Developments}

\newpage
\section{Conclusions}

\newpage
\bibliographystyle{IEEEtran}
\bibliography{ref.bib}

\appendix

Project OneNote \cite{onenote}
Module OneDrive \cite{unidrive}
Private OneDrive \cite{privdrive}

Finalised BoM \cite{bom}
Risk Assessment and COSHH Forms \cite{RA}

GitHub Repository \cite{}

\end{document}