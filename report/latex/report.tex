\documentclass [12pt]{article}
\usepackage{fancyhdr}
\usepackage[utf8]{inputenc}
%\usepackage[english]{babel}
\usepackage{gensymb}
\usepackage{amsmath,amssymb,amsfonts}
\usepackage{algorithmic}
\usepackage{graphicx}
\usepackage{textcomp}
\usepackage[table]{xcolor}
\usepackage{siunitx}
\usepackage{indentfirst}
\usepackage[margin=2.5cm]{geometry}
\usepackage{float}
\usepackage{hyperref}
\usepackage[UKenglish]{datetime}
\usepackage{setspace}
\usepackage{hhline}
\usepackage{lastpage}
\usepackage[acronym]{glossaries}
\usepackage{listings}
\usepackage{color}
\usepackage{booktabs}% http://ctan.org/pkg/booktabs
\newcommand{\tabitem}{~~\llap{\textbullet}~~}
\usepackage{subfigure}
\usepackage[T1]{fontenc}
\usepackage[numbers]{natbib}
\usepackage{helvet}
\renewcommand{\familydefault}{\sfdefault}

\definecolor{customgreen}{rgb}{0,0.6,0}
\definecolor{customgray}{rgb}{0.5,0.5,0.5}
\definecolor{custommauve}{rgb}{0.6,0,0.8}

\makeglossaries
\loadglsentries{glossary.tex}

\graphicspath{{images/}}

\hypersetup
{
    colorlinks=true,
    linkcolor=black,
    urlcolor=blue,
    citecolor=black,
}

\doublespacing

\title{}
\author{}
\date{\today}

\begin{document}

\begin{titlepage}
    \begin{center}
        \vspace*{1cm}

        \vspace*{\baselineskip}

        \rule{\textwidth}{1.6pt}\vspace*{-\baselineskip}\vspace*{2pt}
        \rule{\textwidth}{0.4pt}\\

        {\LARGE EXAMPLE TEXT}\\[0.2\baselineskip]

        \rule{\textwidth}{0.4pt}\vspace*{-\baselineskip}\vspace{3.2pt}
        \rule{\textwidth}{1.6pt}\\[\baselineskip]
        \scshape

        \textbf{Student No: 10618407}

        \today

        \vfill

        \vspace{0.8cm}

        \includegraphics[width=0.4\textwidth]{UOP_Logo.png}

    \end{center}
\end{titlepage}

\newpage
\fancyfoot[R]{Page \thepage \hspace{1pt} of \pageref{LastPage}}
\pagenumbering{roman}
\tableofcontents
\listoffigures
\listoftables
\printglossaries

\newpage
\begin{abstract}

\end{abstract}
\section{Acknowledgements}

\newpage
\pagenumbering{arabic}
\section{Introduction}
\setcounter{page}{1}

\newpage
\section{Literature Review and Prior Art}


\newpage
\section{Business Justifications}

\subsection{Research Techniques and MECE Analysis}

The team engaged in innovation talks throughout the project, particularly with 42Technology \cite{42T}, to understand the framework commonly used to solve complicated problems within industry. A particular method used involved the definition of the problem broken down into mutually exclusive sub-problems that are collectively exhaustive. This method is referred to as MECE Analysis and enabled the team to focus on various elements of the project to achieve one shared goal, without overlapping or reproducing any work. 

This method aided the understanding project criteria from the sponsor’s perspective regarding patenting, and from the university’s perspective of academic evaluation. The PEP defined the market gap, drawing from research gathered that detailed the lack of EV charging infrastructure within the UK, and how realistically, the public cannot rely on shared charging points. However, MECE analysis has uncovered the HEVCS as a feasible solution to such a challenge, and that the situation would benefit from having a mobile, versatile approach to suit the varied requirements of EV users in different locations around the UK. 

\subsection{Addressing the Key Challenges of Electric Vehicle Adoption}

The common barriers preventing the public from purchasing EVs being price, range and charging infrastructure. The HEVCS tackles all three of these problems, and the team have designed the platform and its use specifically for its application within the real-world. 

\subsubsection{Charging Infrastructure and Trends in Usage}

Currently, the UK Government’s National Traffic Survey stated that 938,182 electric cars were registered to be on the roads as of Q3, 2022 \cite{Q32022}. Comparatively, a survey published in October 2022 stated that 34,637 public charging points were currently installed \cite{chargestats}. These figures indicate that the charging infrastructure available cannot keep up with the current number of EV’s on the road and raises concerns as this number is set to rise with the 2030 ban on selling new petrol and diesel vehicles. Furthermore, it is predicated that 99\% of miles will come from EVs by 2050, a figure that has been adjusted after an underestimation of their adoption rate \cite{2050rates}. 

Approximately 50\% of UK charging points are placed at end of journey destinations, typically within car parks the public frequent for shopping, leisure, and down-time activities \cite{chargestats}. A Government survey revealed a majority's preference to charge EVs at home, with 66\% and of BEV and 41\% of PHEV users had their own dedicated charging point within a private driveway or garage \cite{homecharge}. The survey highlights that users felt dedicated charging points were either too expensive or complex to install.

Conversely, 53\% of the UK do not have a driveway or garage that would allow for the possibility of installing a domestic charging point. This has proven to be a deciding factor in the public adoption of EVs. The portion of the market purchasing BEVs/PHEVs using public dedicated charging points within the vicinity of their residence is 4\% and 1\% respectively \cite{chargestats}. Thusly, a market gap to provide a product that can charge EVs within the immediate surroundings of the home has been identified. Moreover, the HEVCS does not require a dedicated parking space, and can be used at the individual's discretion. 

\subsubsection{National Grid Demands}

When considering the typical commuting patterns, full-time workers with typical 9-5 jobs are amongst those susceptible to aligning their arrival to domesticated regions. 60\% of EV users prefer to charge between 5-8pm, meaning the national grid struggles with the demand of those coming home, with up to 20\% of the UK's EV fleet charging between those hours. The HEVCS would charge during the off-peak times, lessening the load on the grid.

\subsubsection{Battery Range and Efficiency}

The National Travel Survey presented statistics in 2021 stating that 98.5\% of journeys taken in the UK were less than 50 miles. The capacity of the payload compartment was calculated through an analysis of both power consumption and efficiency, conducted using data from the UK public’s current EV usage trends. The average efficiency determines that 1 kWh equates to approximately 3.5 miles, and second-hand EV batteries typically have a capacity of 80-90\% of the original value \cite{strickland2014estimation}. The HEVCS’s configurable payload compartment can supply the user with 10 to 50 miles of range, depending on the EV batteries used, as shown in the research and calculations section.

\subsubsection{The Cost of Charging an Electric Vehicle}
The HEVCS presents itself as a cost-effective product that reduces the price of charging an EV. Within the UK, domestic electricity prices have been on the rise, and the cost of charging at public points has increased. Furthermore, the spaces in which to charge are not guaranteed, a free to park still must be paid. A cost analysis presents the additional benefits of an individual having their own EV charging device, such as the HEVCS, that remains domestically powered.

Providers, such as EDF Energy \cite{edf}, offer EV charging tariff that allow the user to charge their vehicle for 9p per kWh during off-peak times, oppose to 46p per kWh at public charging points within Plymouth. The typical UK driver makes 14 trips a week, each with an average distance of 8.4 miles, equating to an annual mileage of \~6,000 miles. This results in a running cost of £788 for vehicles with an efficiency of \~3.5 charging on public tariffs. By using the reduced domestic tariff, the user would spend £154 a year, the savings of which would accumulate to cover the price of the HEVCS platform over a span of several years.

\subsubsection{Environmental Benefits}
The HEVCS can give a second life to EV batteries, while educating users on the importance of understanding their daily commute, and what it demands in terms of power. In addition, it is understood that frequently charging EV batteries to their maximum capacity, and letting them discharge fully, reduces their lifespan. This device can bridge the learning gap, educating users on what it means to operate a battery within its ideal state of charge range to improve longevity. 
It is common knowledge that EV batteries are designed to possess an ability to retain their original battery capacity over many years, but its ability to do so depends on a variety of factors. Charging schedules that keep the state of charge between 20-80\% would provide an adequate amount of power to fulfil the average daily commute. Moreover, the HEVCS would allow the user to take advantage of off-peak electricity rates, smart charging features, and monitoring energy consumption. Through adopting a maintenance charge approach, following the natural patterns to meet the user’s everyday needs, the user can identify how to adjust their habits to positively impact batter health and costs.


\newpage
\section{Requirements}

\newpage
\section{Optioneering}

\newpage
\section{Hardware Design}

\newpage
\section{Software Design}

\newpage
\section{Testing}

\newpage
\section{Evaluation}

\newpage
\section{Further Developments}

\newpage
\section{Conclusions}

\newpage
\bibliographystyle{IEEEtran}
\bibliography{ref.bib}

\appendix

\end{document}