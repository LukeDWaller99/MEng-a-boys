\documentclass [12pt]{article}
\usepackage{fancyhdr}
\usepackage[utf8]{inputenc}
\usepackage[english]{babel}
\usepackage{gensymb}
\usepackage{amsmath,amssymb,amsfonts}
\usepackage{algorithmic}
\usepackage{graphicx}
\usepackage{textcomp}
\usepackage[table]{xcolor}
\usepackage{siunitx}
\usepackage{indentfirst}
\usepackage[margin=2.5cm]{geometry}
\usepackage{float}
\usepackage{hyperref}
\usepackage[UKenglish]{datetime}
\usepackage{setspace}
\usepackage{hhline}
\usepackage{lastpage}
\usepackage[numbers]{natbib}
\usepackage[acronym]{glossaries}
\usepackage{listings}
\usepackage{color}
\usepackage{booktabs}% http://ctan.org/pkg/booktabs
\newcommand{\tabitem}{~~\llap{\textbullet}~~}
\usepackage{subfigure}

\usepackage{helvet}
\renewcommand{\familydefault}{\sfdefault}

\definecolor{customgreen}{rgb}{0,0.6,0}
\definecolor{customgray}{rgb}{0.5,0.5,0.5}
\definecolor{custommauve}{rgb}{0.6,0,0.8}

\graphicspath{{images/}}

\hypersetup
{
    colorlinks=true,
    linkcolor=black,
    urlcolor=blue,
}

\doublespacing

\begin{document}

\title{\bf Report Title}
\author{Daniel Cole, Jack Pendlebury, Noah Harvey, Luke Waller}
\date{\today}
\maketitle
\thispagestyle{empty}

\newpage
\fancyfoot[R]{Page \thepage \hspace{1pt} of \pageref{LastPage}}
\setcounter{page}{1}
\pagenumbering{arabic}
\tableofcontents
\newpage

\listoffigures
\listoftables

\newpage
\section{Section 1}\label{sec:section_1}

\subsection{Subsection 1}\label{sec:sub_section_1}

\cite{PRINCE2}

%\begin{figure}[H]
%\centerline{\includegraphics[width=1\textwidth]{image_1}}
%\caption{Label for Image 1}
%\label{fig:figure_1}
%\end{figure}

\begin{table}[H]
\centering
\setlength{\arrayrulewidth}{1.5pt}
\begin{tabular}{|p{0.3\linewidth}|p{0.3\linewidth}|}
\hline
\cellcolor{gray!40}Component & \cellcolor{gray!40}Device \\
\hline
Controller Microcontroller  & STM32L432KC* \\
\hline
Receiver Microcontroller   & STM32F429ZI\\
\hline
Brushless DC Motors   & A2212 BLDC Motors \\
\hline
BLDC Motor ESC   & Generic no-name 30A ESC\\
\hline
Brushed DC Motor   & Dunkermotoren PLG24 \\
\hline
Brushed DC Motor Driver  & L298N\\
\hline
Controller Batteries   & 18650 Li-Ion Batteries \\
\hline
Buzzer   & Generic no-name buzzer \\
\hline
12V Regulator & 12V Fixed LDO* \\
\hline
5V Regulator & 12V Fixed LDO*\\
\hline
3V Regulator & 12V Fixed LDO*\\
\hline
\multicolumn{2}{l}{\small *selected due to constrains caused by silicon shortage.} \\
\end{tabular}
\caption{Component Selection}
\label{table:component_selection}
\end{table}


{\parindent0pt

Details on how the controller microcontroller software and hardware was developed and interfaced within the scope of the project can be found in  §\ref{sec:} and §\ref{sec:} respectively.

Details on how the receiver microcontroller software and hardware was developed and interfaced within the scope of the project can be found in  §\ref{sec:} and §\ref{sec:} respectively.

Details on how the ESCs were interfaced with can be found in §\ref{sec:}.

Details on how the L298N was interfaced with can be found in §\ref{sec:}.

Detailed on how the 12V, 5V, and 3V Regulators were interfaced with can be found in §\ref{sec:}.

}

\section{Legal Requirements}
\subsection{Legal Restrictions of Use}
The platform will be made to abide class 2 invalid carriage legislation REF HERE, restricting it to a maximum speed of 4mph. It will be compact and as lightweight as possible and compact to accommodate to manoeuvring around homes. Class 2 invalid carriages do not need to be registered with the DVLA.

\subsection{Relevant Legislation}
\subsubsection{Highway Regulations}
The platform must be built to follow the legislation for a class 2 invalid carriage, detailed in the Use of Invalid Carriages \cite{Invalid_Legislation} on Highways Regulations. This device must be mechanically propelled and incapable of exceeding 4mph. The maximum unladen weight of the platform is 113.4kg, where the definition of ‘unladen carriage’ is inclusive of the weight of water, fuel, power equipment for the platform itself and the propulsion equipment. It does not include the weight of any other load, including the weight of the EV batteries that the robot will transport. 

The maximum dimensions of the platform are a product of the hallway, doors, and entrance dimensions within the home, as well as the distance between wheels on the most common EVs. With a maximum height of 155mm, the device would be able to successfully transport a payload underneath 53% of the most common EVs. As class 2 mobility scooters are made to be used indoors, their dimensions determined the maximum length and width is 1000mm and 500mm, respectively. 

\subsubsection{Road Traffic Act 1972}
Other requirements include the functionality of a lights as a motor vehicle would under the Road Traffic Act 1972 \cite{Road_Traffic}. It must also be capable of breaking within a reasonable distance, in all conditions, and remaining stationary on a slope of gradient 1m over a horizontal distance of 5m. The braking system required to hold the platform stationary cannot rely on the limiting of electrical current, hydraulic or pneumatic devices. 

\section{Environmental Analysis}
\subsection{Electric Vehicles}
Electric vehicles are widely toted to be a ‘green’ replacement for vehicles powered by traditional internal-combustion engines. There are, however, environmental concerns at all stages of an EVs lifecycle, which need to be carefully considered by any project seeking to use or benefit EVs. 

The most effective way of comparing vehicle drivetrains on an emissions basis is by measuring the well-to-wheel (WtW) emissions\cite{Well_To_Wheel}. WtW represents a holistic way of comparing drivetrains, by considering the emissions from extraction to consumption. This is a combination of two other measures, well-to-tank (WtT), and tank-to-wheel (TtW). These are the extraction/generation, and the consumption measures, respectively.

 Battery Electric Vehicles (BEVs) emit no TtW emissions, as their electric drivetrain generates no greenhouse gases, however the WtT emissions are over double that of conventional ICE fuels. Even taking this into account, BEVs emit almost three times less CO2 into the atmosphere when compared to conventional ICE fuels. Petrol-powered plug-in hybrids (PHEVs) exist as a middle ground between BEVs and ICE engines, with even higher WtT emissions than BEVs, but WtT emissions are reduced to a third of petrol emissions and 2.5 times lower than diesel. Combined, this puts them between the two drivetrains for total WtT emissions.
 
It should be noted, however, that as of the time of this report, the UK’s energy mix was comprised of significantly higher amounts of energy generated using renewables instead of petroleum products or natural gas. Compared to the EU’s 17\% in 2020\cite{Energy}, the UK is currently generating approx. 45\% of its energy using renewables \cite{Electricity_Generation}. In practice, this would significantly lower the WtT cost of BEV and PHEVs to a point where it would be almost comparable with conventional fuels.

Another benefit of EVs is that these emissions are confined to the source, and so efforts to reduce the CO2/KG emitted can be concentrated on cleaner power generation, or better carbon trapping. An implementation that reduced emissions on EVs would have to be rolled out across an entire production run. Additionally, any EVs already on the road would not benefit from the greener technology unless a manufacturer issued a recall to perform upgrades, an expensive prospect. Given that the largest recall in history was twenty-one million vehicles by Ford in 1980\cite{Ford_Transmission}, and that in 2022 there are already sixteen million EVs\cite{EV_on_Road} on the road worldwide, a large-scale recall could quickly become infeasible.

\subsection{Battery Production}
One of the biggest environmental issues surrounding EVs is in the production of the batteries that power them. Modern battery technology relies on several rare-earth elements, most importantly lithium and cobalt\cite{Lithium_Source}\cite{Earth_Metals}. Lithium extraction has several devastating effects on the environment, effects further exacerbated by the environments that lithium is extracted in\cite{Lithium_Producers}. Lithium extraction is primarily done through the process of evaporation, where subsurface lithium deposits are driven to the surface by pumping water underground\cite{Lithium_Industry}. This water forms a salty brine on the surface, which is then left to evaporate for a prolonged period of time. Lithium carbonate is extracted from the distilled remains of the evaporation process, and this is processed into lithium-based batteries. Over two million litres of water are used per tonne of lithium extracted. Over half the water in Chili’s Atacama Desert is used for lithium extraction, depriving local communities and environments. 

There are alternative methods of lithium extraction, but these have equally devastating effects on the environment. Open pit mining has been widely campaigned against for causing irrevocable harm to local environments and producing substantial amounts of heavy metal-based dust, which is toxic to humans\cite{Open_Pit_Mining}.

Lithium is also a finite resource, with current estimates placing the global supply of lithium at two hundred years’ worth. However, this will go down as production of EVs increases, assuming no non-lithium-based battery technology is adopted\cite{Not_Enough_Lithium}.

A large amount of research and development is going into alternative battery compositions, however lithium-based ones still remain the most cost effective for the energy density received and are the most widely available on the second-hand market.

Battery recycling has made great strides in recent years. A new Volkswagen site in Salzgitter aims to recycle five battery systems per shift, with an annual throughput of 3,600 battery systems for a total output of 1,500 tonnes of recycled material\cite{VW_Press_Release}. Initial expectations are to recover \(>\)70\% of the battery's constituent components by weight, with a target value of \(>\)97\% . Recycling EV batteries to augment global lithium supply would help to offset the increasing demand from the growing fleets of EVs. 

EV battery capacity, represented as state of health (SOH), decreases through the lifetime of the battery in a linear manner that falls rapidly as it approaches end-of-life. This loss averages out to ~2.3\% per year, with the actual rate of SOH decrease affected by both usage and environmental factors\cite{EV_Battery_Health}. Typical EV battery warranties will last for eight years, leading to a source of batteries with a SOH value of ~80\%\cite{EV_Battery_Longevity}. These reduced capacity batteries are unsuitable for automotive purposes but have potential applications in use-cases that require less arduous capacity requirements.

\subsection{Chassis Manufacture}
Aluminium extrusion will be used to produce the prototype chassis. Aluminium is mined as bauxite ore, which is refined into alumina, from which aluminium is produced. However, the manufacture of aluminium extrusion is, compared to some other metal products, environmentally friendly. Over 75\% of the aluminium produced since 1888 is still in use today, and over 90\% of the energy used in European aluminium smelting is from zero-carbon sources.

The production of primary aluminium, that is, aluminium produced directly from alumina, has steadily decreased in its environmental impact. In 1995, producing one tonne of aluminium produced 16.5 tonnes of C02 equivalent. Updated processes dropped that to 12 tonnes of CO2 equivalent in 2018, and as of 2022, multiple aluminium producers have product available with a CO2 equivalent of less than 4 tonnes available\cite{Al_Manufac}.

By using aluminium instead of comparative materials such as steel, HEVCS hopes to lower its environmental impact. Once the prototype is no longer needed, the chassis can be disassembled to be recycled, with pieces of extrusion either re-used for other purposes, or sent to be recycled and re-made into new aluminium products. 


\newpage
\bibliographystyle{IEEEtranN}
\bibliography{refs}

\newpage
\appendix

\section{Appendix 1}\label{app:appendix_1}


\end{document}