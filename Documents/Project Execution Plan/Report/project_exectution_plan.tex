\documentclass [12pt]{article}
\usepackage{fancyhdr}
\usepackage[utf8]{inputenc}
\usepackage[english]{babel}
\usepackage{gensymb}
\usepackage{amsmath,amssymb,amsfonts}
\usepackage{algorithmic}
\usepackage{graphicx}
\usepackage{textcomp}
\usepackage[table]{xcolor}
\usepackage{siunitx}
\usepackage{indentfirst}
\usepackage[margin=2.5cm]{geometry}
\usepackage{float}
\usepackage{hyperref}
\usepackage[UKenglish]{datetime}
\usepackage{setspace}
\usepackage{hhline}
\usepackage{lastpage}
\usepackage[numbers]{natbib}
\usepackage[acronym]{glossaries}
\usepackage{listings}
\usepackage{color}
\usepackage{booktabs}% http://ctan.org/pkg/booktabs
\newcommand{\tabitem}{~~\llap{\textbullet}~~}
\usepackage{subfigure}

\usepackage{helvet}
\renewcommand{\familydefault}{\sfdefault}

\definecolor{customgreen}{rgb}{0,0.6,0}
\definecolor{customgray}{rgb}{0.5,0.5,0.5}
\definecolor{custommauve}{rgb}{0.6,0,0.8}

\graphicspath{{images/}}

\hypersetup
{
    colorlinks=true,
    linkcolor=black,
    urlcolor=blue,
}

\doublespacing

\begin{document}

\title{\bf Report Title}
\author{Daniel Cole, Jack Pendlebury, Noah Harvey, Luke Waller}
\date{\today}
\maketitle
\thispagestyle{empty}

\newpage
\fancyfoot[R]{Page \thepage \hspace{1pt} of \pageref{LastPage}}
\setcounter{page}{1}
\pagenumbering{arabic}
\tableofcontents
\newpage

\listoffigures
\listoftables

\newpage

\section{Health and Safety}\label{sec:health_and_safety}
\subsection{Risk Assessment}\label{sec:risk_assessment}

A Risk Assessment has been conducted to critically evaluate the risks involved with the project's operation. This includes their severity, and the best way to mitigate them. 
The Risk Assessment is a useful document for tracking and monitoring the risks of the project. 
This document serves as a safe practises guide for operating dangerous tools and machinery.


For the Risk Assessment, refer to the separate document.

\subsection{COSHH Forms}\label{sec:coshh_forms}

A series of control of substances hazardous to health (COSHH) forms have been completed in accordance with manufacturers recommendations. 
The COSHH forms outline the safe usage for substances that are potentially damaging to both the user's health and the environment. 
These forms are a useful aid for tracking and monitoring the potentially harmful substances used within the project. 
These documents serve as safe practise guides for working with dangerous substances.


For the COSHH forms, please refer to the separate documents.

\subsection{Saftey Measures}\label{sec:safety_measures}

Given the large form factor and heavy payload requirements of the HEVCS platform, following appropriate safety measures and protocols will be vital in ensuring the wellbeing of the team members. 
Before beginning any hardware development, all participating members will be required to read the relevant risk assessments and COSHH forms for the activity they are undertaking. 
This will ensure the safest possible working environment in which team members are aware of the risks of any given activity, and how to mitigate against them. 
If these protocols are not followed it could lead to potentially life altering or fatal accidents. 
As a result of this, multiple safety measures have been considered and implemented. 
These have been outlined below.


Safety measures involve collision avoidance by using LiDAR sensors that detect obstacles around the platform, activating the brakes before impact. 
The motors will also be stopped if the platform is trying to be driven down a slope greater than 20 degrees or a set of steps with a larger  than specified appropriate drop. 
Thresholds will be set to stop the platform from traversing terrain it is not capable of.


The controller for the platform will utilise a breakaway connector which disconnects from the platform when it moves too far from the operator, causing the platform to halt operation, immobilising the motors. 
The platform will also enter an immobilised state of operation if the controller becomes unresponsive at any point.


When testing the platform, a safe environment will be utilised. Only persons required to be in attendance will be allowed within the testing environment. 
For full scale testing of a new firmware version uploaded to the platform, only the team members are allowed to be within the testing environment. 
For finalised testing, the platform will be subjected to all environments that it is likely to experience in daily use, such as dark roads and uneven flooring.


To reduce the risk of a bug in firmware being uploaded to the full-scale platform, a series of small-scale mock-ups and testing rigs will be made and utilised.


Safe coding practises will be observed when writing and uploading code onto the platform. 
These include changes to the code being checked, and a testing procedure being run before approval to upload the code is given. 
This will implement a three-step code approval process starting with a test bench, continued into a small-scale model before being uploaded onto the full-sized platform. 
Before changes to the code are uploaded onto the full-sized platform, the changes must be reviewed and approved by another member of the group.


Any safety critical sections of code will require the previously outlined levels of approval with further safety measures being implemented. 
The safety critical code must be checked and approved by all members of the group. 
This will involve a full review by the team including an outline of the changes presented by the group member who implemented them.
Safety critical elements of the code are defined as any code that modifies the locomotion of the platform or adapts how the platform behaves when the controller is disconnected. 

\section{Internationalisation - Global Market}\label{sec:internationalisation}

Introducing the HEVCS to the global market is highly achievable but would need to be done with several considerations in mind.
Given the nature of charging batteries from within the home, suitable infrastructure would need to be evaluated.
Given that the UK runs off 230V 50Hz AC power, the HEVCS will be designed to run off these ratings. 
This would work throughout Europe, due to the voltages being harmonised in 2003 (when the UK was still part of the European Union (EU))\cite{UK_Voltage_Change}, meaning that only the plug connected to the platform would need to be changed. 
Using a universal plug connection, such as an IEC 60320, would allow the charging circuitry to be universal in any country that uses the same power levels as the UK.
Other locations, such as the United States of America (USA), use 120Vac so would require differing power circuit to be implemented\cite{Voltages_And_Frequencies}. 


Along with charging circuitry, the HEVCS platform would need to be checked against the plurality of local legislation. 
Local regulations for step height, doorframe size, and curb dimensions, would need to be checked and if required, the design may need to be adapted to accommodate global regulations. 


The platforms classification may also need to be adapted, depending on the specific country requirements. 
This could be anything from adapting the size of the platform, to reducing its payload. 
Some of these may require a considerable redesign to accommodate regional legislation.
\newpage
\bibliographystyle{IEEEtranN}
\bibliography{refs}

\newpage
\appendix

\section{Appendix 1}\label{app:appendix_1}


\end{document}