\documentclass [12pt]{article}
\usepackage{fancyhdr}
\usepackage[utf8]{inputenc}
\usepackage[english]{babel}
\usepackage{gensymb}
\usepackage{amsmath,amssymb,amsfonts}
\usepackage{algorithmic}
\usepackage{graphicx}
\usepackage{textcomp}
\usepackage[table]{xcolor}
\usepackage{siunitx}
\usepackage{indentfirst}
\usepackage[margin=2.5cm]{geometry}
\usepackage{float}
\usepackage{hyperref}
\usepackage[UKenglish]{datetime}
\usepackage{setspace}
\usepackage{hhline}
\usepackage{lastpage}
\usepackage[numbers]{natbib}
\usepackage[acronym]{glossaries}
\usepackage{listings}
\usepackage{color}
\usepackage{booktabs}% http://ctan.org/pkg/booktabs
\newcommand{\tabitem}{~~\llap{\textbullet}~~}
\usepackage{subfigure}

\usepackage{helvet}
\renewcommand{\familydefault}{\sfdefault}

\definecolor{customgreen}{rgb}{0,0.6,0}
\definecolor{customgray}{rgb}{0.5,0.5,0.5}
\definecolor{custommauve}{rgb}{0.6,0,0.8}

\graphicspath{{images/}}

\hypersetup
{
    colorlinks=true,
    linkcolor=black,
    urlcolor=blue,
}

\doublespacing

\begin{document}

\title{\bf Report Title}
\author{Daniel Cole, Jack Pendlebury, Noah Harvey, Luke Waller}
\date{\today}
\maketitle
\thispagestyle{empty}

\newpage
\fancyfoot[R]{Page \thepage \hspace{1pt} of \pageref{LastPage}}
\setcounter{page}{1}
\pagenumbering{arabic}
\tableofcontents
\newpage

\listoffigures
\listoftables

\newpage
\section{Section 1}\label{sec:section_1}

\subsection{Subsection 1}\label{sec:sub_section_1}

\cite{PRINCE2}

%\begin{figure}[H]
%\centerline{\includegraphics[width=1\textwidth]{image_1}}
%\caption{Label for Image 1}
%\label{fig:figure_1}
%\end{figure}

\begin{table}[H]
\centering
\setlength{\arrayrulewidth}{1.5pt}
\begin{tabular}{|p{0.3\linewidth}|p{0.3\linewidth}|}
\hline
\cellcolor{gray!40}Component & \cellcolor{gray!40}Device \\
\hline
Controller Microcontroller  & STM32L432KC* \\
\hline
Receiver Microcontroller   & STM32F429ZI\\
\hline
Brushless DC Motors   & A2212 BLDC Motors \\
\hline
BLDC Motor ESC   & Generic no-name 30A ESC\\
\hline
Brushed DC Motor   & Dunkermotoren PLG24 \\
\hline
Brushed DC Motor Driver  & L298N\\
\hline
Controller Batteries   & 18650 Li-Ion Batteries \\
\hline
Buzzer   & Generic no-name buzzer \\
\hline
12V Regulator & 12V Fixed LDO* \\
\hline
5V Regulator & 12V Fixed LDO*\\
\hline
3V Regulator & 12V Fixed LDO*\\
\hline
\multicolumn{2}{l}{\small *selected due to constrains caused by silicon shortage.} \\
\end{tabular}
\caption{Component Selection}
\label{table:component_selection}
\end{table}


{\parindent0pt

Details on how the controller microcontroller software and hardware was developed and interfaced within the scope of the project can be found in  §\ref{sec:} and §\ref{sec:} respectively.

Details on how the receiver microcontroller software and hardware was developed and interfaced within the scope of the project can be found in  §\ref{sec:} and §\ref{sec:} respectively.

Details on how the ESCs were interfaced with can be found in §\ref{sec:}.

Details on how the L298N was interfaced with can be found in §\ref{sec:}.

Detailed on how the 12V, 5V, and 3V Regulators were interfaced with can be found in §\ref{sec:}.

}

\section{Legal Requirements}
\subsection{Legal Restrictions of Use}
The platform will be made to abide class 2 invalid carriage legislation REF HERE, restricting it to a maximum speed of 4mph. It will be compact and as lightweight as possible and compact to accommodate to manoeuvring around homes. Class 2 invalid carriages do not need to be registered with the DVLA.

\subsection{Relevant Legislation}
\subsubsection{Highway Regulations}
The platform must be built to follow the legislation for a class 2 invalid carriage, detailed in the Use of Invalid Carriages \cite{Invalid_Legislation} on Highways Regulations. This device must be mechanically propelled and incapable of exceeding 4mph. The maximum unladen weight of the platform is 113.4kg, where the definition of ‘unladen carriage’ is inclusive of the weight of water, fuel, power equipment for the platform itself and the propulsion equipment. It does not include the weight of any other load, including the weight of the EV batteries that the robot will transport. 

The maximum dimensions of the platform are a product of the hallway, doors, and entrance dimensions within the home, as well as the distance between wheels on the most common EVs. With a maximum height of 155mm, the device would be able to successfully transport a payload underneath 53% of the most common EVs. As class 2 mobility scooters are made to be used indoors, their dimensions determined the maximum length and width is 1000mm and 500mm, respectively. 

\subsubsection{Road Traffic Act 1972}
Other requirements include the functionality of a lights as a motor vehicle would under the Road Traffic Act 1972 \cite{Road_Traffic}. It must also be capable of breaking within a reasonable distance, in all conditions, and remaining stationary on a slope of gradient 1m over a horizontal distance of 5m. The braking system required to hold the platform stationary cannot rely on the limiting of electrical current, hydraulic or pneumatic devices. 

\newpage
\bibliographystyle{IEEEtranN}
\bibliography{refs}

\newpage
\appendix

\section{Appendix 1}\label{app:appendix_1}


\end{document}