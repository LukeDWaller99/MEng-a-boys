\documentclass [12pt]{article}
\usepackage{fancyhdr}
\usepackage[utf8]{inputenc}
\usepackage[english]{babel}
\usepackage{gensymb}
\usepackage{amsmath,amssymb,amsfonts}
\usepackage{algorithmic}
\usepackage{graphicx}
\usepackage{textcomp}
\usepackage[table]{xcolor}
\usepackage{siunitx}
\usepackage{indentfirst}
\usepackage[margin=2.5cm]{geometry}
\usepackage{float}
\usepackage{hyperref}
\usepackage[UKenglish]{datetime}
\usepackage{setspace}
\usepackage{hhline}
\usepackage{lastpage}
\usepackage[numbers]{natbib}
\usepackage[acronym]{glossaries}
\usepackage{listings}
\usepackage{color}
\usepackage{booktabs}% http://ctan.org/pkg/booktabs
\newcommand{\tabitem}{~~\llap{\textbullet}~~}
\usepackage{subfigure}

\usepackage{helvet}
\renewcommand{\familydefault}{\sfdefault}

\definecolor{customgreen}{rgb}{0,0.6,0}
\definecolor{customgray}{rgb}{0.5,0.5,0.5}
\definecolor{custommauve}{rgb}{0.6,0,0.8}

\graphicspath{{images/}}

\hypersetup
{
    colorlinks=true,
    linkcolor=black,
    urlcolor=blue,
}

\doublespacing

\begin{document}

\title{\bf IP Landscape}
\author{Daniel Cole, Jack Pendlebury, Noah Harvey, Luke Waller}
\date{\today}
\maketitle
\thispagestyle{empty}

\newpage
\fancyfoot[R]{Page \thepage \hspace{1pt} of \pageref{LastPage}}
\setcounter{page}{1}
\pagenumbering{arabic}
\tableofcontents
\newpage

\listoffigures
\listoftables

%\begin{figure}[H]
%\centerline{\includegraphics[width=1\textwidth]{image_1}}
%\caption{Label for Image 1}
%\label{fig:figure_1}
%\end{figure}

% §\ref{sec:section} to reference a section
% \ref{app:appendix} to reference an appendix
% \cite{} to refer to bib

\newpage
\section{Patent Landscape Analysis}\label{sec:PLA}

\section{Project Overview}\label{sec:project_overview}
\subsection{Project Details}\label{sec:project_details}
\subsection{Inspiration}\label{sec:inspiration}
\subsection{Intended Market}\label{sec:intended_market}
\subsection{Safety and Ethical Considerations}\label{sec:SAEC}
\subsection{Market Competition}\label{sec:market_competition}
\subsection{Design Requirements}\label{sec:design_requirements}

\section{HEVCS Key Constituent Parts}\label{sec:constituent_parts}

\section{Patent Research}\label{sec:patent_research}
\subsection{Under Vehicle Technology}\label{sec:under_vehilce_technology}
\subsection{Breakaway Connectors}\label{sec:breakaway_connectors}
\subsubsection{Search Strategies}
\textbf{Strategy 1}\\
Using the advanced search functionality of espacenet to filter results based on a search terms inclusion in just the title, the phrase “magnetic” and “break away” was used and returned a single patent. 
Similar combinations with less strict filters returned more results, with “parallel” AND “cable” AND “magnetic” returning 29 results. \\
\textbf{Strategy 2}\\
An alternate strategy was attempted, by placing emphasis on the break-away/pull-apart functionality of the connector. This led to the following results:
\begin{table}[H]
    \centering
    \setlength{\arrayrulewidth}{1.5pt}
    \begin{tabular}{|p{0.7\linewidth}|c|}
    \hline
    \cellcolor{gray!40}Search Term & \cellcolor{gray!40}No. of Results \\
    \hline
    ti = “magnetic” AND (ti any “break away” OR ti any “pull apart”) & 508 \\
    \hline
    ti = “magnetic” AND (ti any “break away” OR ti any “pull apart”) AND ti = “Connector” & 5 \\
    \hline
    ti = “magnetic” AND (ti any “break away” OR ti any “pull apart”) AND ti = “Connector” AND (ctxt = “electrical” OR ctxt = “electronic”) & 2 \\
    \hline
    \end{tabular}
    \caption{Breakaway Connector Search Terms 1}
    \label{table:breakaway_connector_search_strat_2}
\end{table}
Of the two patents found at the conclusion of this search strategy, only one had relevance to the project. Despite use of logic OR statements to look for synonyms and word variants, this search strategy was overly specialised and too narrow. To remedy this, a third strategy was employed.\\

\textbf{Strategy 3}
\begin{table}[H]
    \centering
    \setlength{\arrayrulewidth}{1.5pt}
    \begin{tabular}{|p{0.7\linewidth}|c|}
    \hline
    \cellcolor{gray!40}Search Term & \cellcolor{gray!40}No. of Results \\
    \hline
    ti = “magnetic” AND ti = “Connector” AND (ti = “electrical” OR ti = “electronic”) & 149 \\
    \hline
    ti = “magnetic” AND (ti = “Connector” OR ti = “Socket”) AND ti = “parallel” & 2 \\
    \hline
    ti = “magnetic” AND (ti = “Connector” OR ti = “Socket”) AND (ti = “parallel” OR ti = “data”) & 25 \\
    \hline
    (ti = “magnetic” AND (ti = “Connector” OR ti = “Socket”) AND (ti = “parallel” OR ti = “data”)) NOT ctxt = “power” & 22 \\
    \hline
    (ti = “magnetic” AND (ti = “Connector” OR ti = “Socket”) AND (ti = “parallel” OR ti = “data”)) NOT (ctxt = “power” OR ctxt = “USB”) & 16\\
    \hline
    \end{tabular}
    \caption{Breakaway Connector Search Terms 1}
    \label{table:breakaway_connector_search_strat_3}
\end{table}
\subsubsection{Patent 1 - Variable Magnetic Break-away Mounting Mechanism}
\begin{table}[H]
    \centering
    \setlength{\arrayrulewidth}{1.5pt}
    \begin{tabular}{|p{0.5\linewidth}|p{0.5\linewidth}|}
    \hline
    Patent Number & \href{https://worldwide.espacenet.com/patent/search/family/054932262/publication/US9221397B1?q=US9221397B1}{US9221397B1}\\
    \hline
    Applicants & Google Inc [US]\\
    \hline
    Status & Active\\
    \hline
    Application Date/Publish Date & 2014-10-29 / 2014-05-07\\
    \hline
    Active Jurisdictions & US\\
    \hline
    \end{tabular}
    \caption{Patent information}
    \label{table:mag_con1}
\end{table}
\textbf{Claim 1}\\
Claim 1 describes a magnetic coupling system that utilises two round connectors with a magnetic element in the centre. One connector is concave, the other is convex, and so the two halves fit together, held in-place by the magnet. Each half can spin independently of the other, as the single magnet provides no rotational locking.
Whilst the magnetic break away connector that HEVCS will employ shares the same male/female aspect with a single magnetic locking pin, the HEVCS connector will not share the rotational symmetry that the connector specified in the patent does. Additionally, the patent describes a mechanical coupling device. The HEVCS connector will be connecting the control unit to the mobile platform by way of a 16-pin parallel cable, differing in that it will be a data cable and not a purely mechanical coupling.

\textbf{Claim 2}\\
Claim 18 is identical in content to Claim 1, and so for the reasons outlined above, will not be a factor in infringement.

\textbf{Summary}\\
In summary, the HEVCS design will not infringe this patent, as it does not violate either of the two independent claims. It should be noted, however, that while this patent is valid only in the US, it is active and being upheld by a major US corporation, and so would be a high risk for litigation should infringement occur. 

\subsubsection{Patent 2 - Magnetic Connector for a Data Communications Cable}
\begin{table}[H]
    \centering
    \setlength{\arrayrulewidth}{1.5pt}
    \begin{tabular}{|p{0.5\linewidth}|p{0.5\linewidth}|}
    \hline
    Patent Number & \href{https://worldwide.espacenet.com/patent/search/family/042731083/publication/US10855023B2?q=pn%3DUS10855023B2}{US10855023B2}\\
    \hline
    Applicants & MASIMO CORP [US]\\
    \hline
    Status & Active\\
    \hline
    Application Date/Publish Date & 2018-12-21 / 2020-12-01\\
    \hline
    Active Jurisdictions & US\\
    \hline
    \end{tabular}
    \caption{Patent information}
    \label{table:mag_con2}
\end{table}

\textbf{Claim 1}\\
Claim 1 describes a matching pair of magnetic connectors for a low-voltage data transmission cable. The connectors will use a toggleable electromagnetic coupling system in order to align the connectors and ensure constant connectivity, as well as allowing for intentional disconnection by deactivating the electromagnets.
HEVCS’ connector design will use magnets for alignment and maintaining connection, however, these will be standard always-on magnets, lacking the toggle on/off functionality that the claim outlines. The HEVCS connector uses the magnets to provide a baseline holding force, that in a runaway event will come disconnected without injuring the user. These cannot be de/remagnetised at will, unlike electromagnets, thus this claim is not infringed. 
\textbf{Summary}\\
Despite apparent similarities in the use of magnetic forces to hold a data connector in place, this patent is not infringed due to the differing methods of implementation. Being granted in 2020, this is a recent patent held by a major corporate entity, and so infringement should be expected to result in litigation. However, as the only active jurisdiction is the US, infringement is not a concern at this time.
Masimo Corporation produce medical technology products, and this specific patent is for use in a medical pulse oximeter. This has no need for the enhanced safety features built into the HEVCS’ connector, the disconnect at will feature being much more important to this specific application.

\subsection{Curb and Step Navigation}\label{sec:curb_step_navigation}
\subsubsection{Patent 1 - Stair traversing device}

\begin{table}[H]
    \centering
    \setlength{\arrayrulewidth}{1.5pt}
    \begin{tabular}{|p{0.5\linewidth}|p{0.5\linewidth}|}
    \hline
    Patent Number & CN108349516A\\
    \hline
    Applicants & QUANTUM ROBOTIC SYSTEMS INC\\
    \hline
    Status & Active\\
    \hline
    Application Date/Publish Date & 2018-07-31 / 2021-05-11\\
    \hline
    Active Jurisdictions & CN\\
    \hline
    \end{tabular}
    \caption{Stair traversing device patent information}
    \label{table:stair_traversing_device_patent_information}
\end{table}

URL - \url{https://worldwide.espacenet.com/patent/search/family/058762787/publication/CN108349516A?q=CN108349516A}

\begin{table}[H]
    \centering
    \setlength{\arrayrulewidth}{1.5pt}
    \begin{tabular}{|p{0.7\linewidth}|c|}
    \hline
    \cellcolor{gray!40}Search Term & \cellcolor{gray!40}No. of Results \\
    \hline
    nftxt = "carriage" AND nftxt = "Stairs" & 7038 \\
    \hline
    nftxt = "carriage" AND nftxt = "Stairs" AND nftxt = "assisted" & 290 \\
    \hline
    nftxt = "carriage" AND nftxt = "Stairs" AND nftxt = "assisted" AND nftxt = "climbing" & 117 \\
    \hline
    "Stairs" AND nftxt = "assisted" AND nftxt = "climbing" AND nftxt = "remote controlled" & 8 \\
    \hline
    \end{tabular}
    \caption{Step/Curb Navigation Search Terms 1}
    \label{table:step_curb_nav_st_1}
\end{table}

\textbf{Claims}

A device for climbing stairs comprising: A carrier body for transporting a payload,Ladder Frame, A mechanism between the stepped frame and the carrier body, The mechanism is configured to move the stepped frame in relation to the carrier body in a circular path.

The HEVCS does not contain two individual parts that move in relation to each other in a circular motion. This means that this patent is not relevant to the HEVCS platform and there does not impede the patent.

The stair climbing device comprises of a main body portion and an outer hanging portion, forming an L shaped design.
The HEVCS is not an L shaped design comprising of two exactly different body portions. Therefore, the HEVCS platform does not encroach on this patent.

\subsubsection{Patent 2 - Improvements in or relating to first/final mile transportation}

\begin{table}[H]
    \centering
    \setlength{\arrayrulewidth}{1.5pt}
    \begin{tabular}{|p{0.5\linewidth}|p{0.5\linewidth}|}
    \hline
    Patent Number & CN108069182A\\
    \hline
    Applicants & FORD GLOBAL TECH LLC\\
    \hline
    Status & Active\\
    \hline
    Application Date/Publish Date & 2018-05-25 / 2022-03-11\\
    \hline
    Active Jurisdictions & CN, USA, DN, UK\\
    \hline
    \end{tabular}
    \caption{Improvements in or relating to first/final mile transportation patent information}
    \label{table:improvements_in_or_relating_to_first_final_mile_transportation_patent_information}
\end{table}

URL - \url{https://worldwide.espacenet.com/patent/search/family/062016996/publication/CN108069182A?q=CN108069182A}

\begin{table}[H]
    \centering
    \setlength{\arrayrulewidth}{1.5pt}
    \begin{tabular}{|p{0.7\linewidth}|c|}
    \hline
    \cellcolor{gray!40}Search Term & \cellcolor{gray!40}No. of Results \\
    \hline
    nftxt = "Curb Climbing" & 161 \\
    \hline
    nftxt = "Curb climbing" AND nftxt = "assisted" & 20 \\
    \hline
    nftxt = "Curb climbing" AND nftxt = "Stairs" AND nftxt = "assisted" & 11 \\
    \hline
    nftxt = "Curb climbing" AND nftxt = "step climbing" AND nftxt = "Stairs" AND nftxt = "assisted" AND nftxt = "climbing" & 3 \\
    \hline
    \end{tabular}
    \caption{Step/Curb Navigation Search Terms 2}
    \label{table:step_curb_nav_st_2}
\end{table}

\textbf{Claims}

A device for transporting payloads over varying terrain, the device comprises: a first round of clustering of three wheels contained in a flat configuration, a second round of clusters of three wheels contained in a planar configuration.
The HEVCS will not use a cluster of three wheels to allow the platform to move up and down stairs. Therefore, this does not encroach on this patent.
Given that this claim only has one independent claim, outlining the use of three clustered wheels, this patent is not being impeded by the HEVCS platform.

\subsubsection{Patent 3 - ROBOTIC VEHICLE}

\begin{table}[H]
    \centering
    \setlength{\arrayrulewidth}{1.5pt}
    \begin{tabular}{|p{0.5\linewidth}|p{0.5\linewidth}|}
    \hline
    Patent Number & WO2011152890A2\\
    \hline
    Applicants & IROBOT CORP\\
    \hline
    Status & Active\\
    \hline
    Application Date/Publish Date & 2010-09-23 / 2013-03-20\\
    \hline
    Active Jurisdictions & USA, WIPO\\
    \hline
    \end{tabular}
    \caption{ROBOTIC VEHICLE patent information}
    \label{table:robotic_vehicle_patent_information}
\end{table}

URL - \url{https://worldwide.espacenet.com/patent/search/family/044675803/publication/WO2011152890A2?q=WO2011152890A2}

\begin{table}[H]
    \centering
    \setlength{\arrayrulewidth}{1.5pt}
    \begin{tabular}{|p{0.7\linewidth}|c|}
    \hline
    \cellcolor{gray!40}Search Term & \cellcolor{gray!40}No. of Results \\
    \hline
    nftxt = "Stair climbing" & 7688 \\
    \hline
    nftxt = "Stair climbing" AND nftxt = "assisted" & 788 \\
    \hline
    nftxt = "Wheeled" AND nftxt = "Stair Climbing" AND nftxt = "assisted" & 107 \\
    \hline
    nftxt = "Stair climbing" AND nftxt = "assisted" AND nftxt = "platform" AND nftxt = "wheeled" AND nftxt = "remote controlled"& 6 \\
    \hline
    \end{tabular}
    \caption{Step/Curb Navigation Search Terms 3}
    \label{table:step_curb_nav_st_3}
\end{table}

\textbf{Claims}

A Robotic Device comprising a chassis having front and rear ends and supported on each right and left driven tracks.
The HEVCS platform will not be driven using tracks, therefore is not in breach of the Patent.
A deck assembly configured to receiver a removeable payload; and a linkage connecting the deck assembly to the chassis.
The HEVCS platform will house the payload within the main chassis and not on a deck connected to a linkage. Therefor this does not impeach on the outlined patent.
This means that the HEVCS platform is not into breech of this patent.

\subsubsection{Patent 4 - Conveying mechanism for grandstand seat area}

\begin{table}[H]
    \centering
    \setlength{\arrayrulewidth}{1.5pt}
    \begin{tabular}{|p{0.5\linewidth}|p{0.5\linewidth}|}
    \hline
    Patent Number & CN111976576A\\
    \hline
    Applicants & WANG JIANPING\\
    \hline
    Status & Application Withdrawn\\
    \hline
    Application Date/Publish Date & 2020-11-24 / 2022-10-21\\
    \hline
    Active Jurisdictions & Not Active\\
    \hline
    \end{tabular}
    \caption{ROBOTIC VEHICLE patent information}
    \label{table:robotic_vehicle_patent_information}
\end{table}

URL - \url{https://worldwide.espacenet.com/patent/search/family/044675803/publication/WO2011152890A2?q=WO2011152890A2}

\begin{table}[H]
    \centering
    \setlength{\arrayrulewidth}{1.5pt}
    \begin{tabular}{|p{0.7\linewidth}|c|}
    \hline
    \cellcolor{gray!40}Search Term & \cellcolor{gray!40}No. of Results \\
    \hline
    nftxt = "Stair climbing" & 7688 \\
    \hline
    nftxt = "Stair climbing" AND nftxt = "assisted" & 788 \\
    \hline
    nftxt = "Wheeled" AND nftxt = "Stair Climbing" AND nftxt = "assisted" & 107 \\
    \hline
    nftxt = "Stair climbing" AND nftxt = "assisted" AND nftxt = "platform" AND nftxt = "wheeled" AND nftxt = "remote controlled"& 6 \\
    \hline
    \end{tabular}
    \caption{Step/Curb Navigation Search Terms 3}
    \label{table:step_curb_nav_st_3}
\end{table}

\textbf{Claims}

A Robotic Device comprising a chassis having front and rear ends and supported on each right and left driven tracks.
The HEVCS platform will not be driven using tracks, therefore is not in breach of the Patent.
A deck assembly configured to receiver a removeable payload; and a linkage connecting the deck assembly to the chassis.
The HEVCS platform will house the payload within the main chassis and not on a deck connected to a linkage. Therefor this does not impeach on the outlined patent.
This means that the HEVCS platform is not into breech of this patent.

\subsubsection{Patent 5 - REMOTE-OPERATED MULTI-DIRECTIONAL TRANSPORT VEHICLE}

\begin{table}[H]
    \centering
    \setlength{\arrayrulewidth}{1.5pt}
    \begin{tabular}{|p{0.5\linewidth}|p{0.5\linewidth}|}
    \hline
    Patent Number & WO0246031A1\\
    \hline
    Applicants & ALLARD ERIC J\\
    \hline
    Status & Patent Expired\\
    \hline
    Application Date/National Phase & 2002-06-13 / 2004-09-08\\
    \hline
    Active Jurisdictions & Patent Expired\\
    \hline
    \end{tabular}
    \caption{REMOTE-OPERATED MULTI-DIRECTIONAL TRANSPORT VEHICLE}
    \label{table:remote_operated_multi_directional_transport_vehicle_patent_information}
\end{table}

URL - \url{https://worldwide.espacenet.com/patent/search/family/021742047/publication/WO0246031A1?q=WO0246031A1}

\begin{table}[H]
    \centering
    \setlength{\arrayrulewidth}{1.5pt}
    \begin{tabular}{|p{0.7\linewidth}|c|}
    \hline
    \cellcolor{gray!40}Search Term & \cellcolor{gray!40}No. of Results \\
    \hline
    nftxt = "Stair climbing" OR nftxt = "Curb Climbing" & 7826 \\
    \hline
    (nftxt = "Stair climbing" OR nftxt = "Curb Climbing") AND (nftxt = "assisted" OR nftxt = "assisting") & 1331 \\
    \hline
    (nftxt = "Stair climbing" OR nftxt = "Curb Climbing") AND (nftxt = "assisted" OR nftxt = "assisting") AND (nftxt = "platform" OR nftxt = "carriage") & 396 \\
    \hline
    (nftxt = "Stair climbing" OR nftxt = "Curb Climbing") AND (nftxt = "assisted" OR nftxt = "assisting") AND (nftxt = "platform" OR nftxt = "carriage") AND nftxt = "wheeled" AND nftxt = "remote controlled"& 7 \\
    \hline
    \end{tabular}
    \caption{Step/Curb Navigation Search Terms 5}
    \label{table:step_curb_nav_st_5}
\end{table}

\textbf{Claims}

A chassis containing a pair of laterally opposed front axles, a pair of lateral opposed rear axles and a pair of longitudinal intermediate axels between the front and rear axles.
Given the HEVCS will not have longitudinally mounted axels and will drive by one motor on each of the wheels, the patent is not impeded by the platform.
This patent is also expired meaning that even if the HEVCS platform did infringe on the patent it would not be affeced.



\subsection{EV Charging}\label{sec:ev_charging}

\cite{Home_Chargepoints}

\subsubsection{Search Strategy 1}\label{sec:EV_Search1}




 \newpage
 \bibliographystyle{IEEEtranN}
 \bibliography{refs}

\newpage
\appendix

\section{Appendix 1}\label{app:appendix_1}


\end{document}